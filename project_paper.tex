\documentclass[conference]{IEEEtran}
\usepackage{cite}
\usepackage{amsmath,amssymb,amsfonts}
\usepackage{algorithmic}
\usepackage{graphicx}
\usepackage{textcomp}
\usepackage{xcolor}
\usepackage{hyperref}

\begin{document}

\title{AI-Powered Automated Form Filling System using Large Language Models}

\author{\IEEEauthorblockN{Your Name}
\IEEEauthorblockA{\textit{Department of Computer Science} \\
\textit{University Name}\\
City, Country \\
email@email.com}
}

\maketitle

\begin{abstract}
The manual process of filling out web forms, particularly for job applications and bureaucracy, is time-consuming and prone to redundancy. Existing autofill solutions often fail to handle complex, open-ended questions or context-specific requirements. This paper proposes an intelligent form filling system that leverages Large Language Models (LLMs) to automate data entry. By integrating a browser extension with a FastAPI backend and the Groq inference engine, our system achieves context-aware processing of form fields. We demonstrate that this architecture significantly reduces the time required for form completion while maintaining high accuracy and personalization.
\end{abstract}

\begin{IEEEkeywords}
Large Language Models, Browser Extension, Automation, Groq API, FastAPI, DevOps.
\end{IEEEkeywords}

\section{Introduction}
As web services proliferate, users are increasingly burdened with repetitive data entry tasks. While browser-based autofill features perform well for standard fields (e.g., name, address), they lack the semantic understanding required for unstructured inputs, such as "Describe a challenge you overcame."

Deep learning and specifically Large Language Models (LLMs) offer a solution by enabling systems to understand context and generate human-like text. This project, \textit{AI Form Filler}, utilizes the high-speed inference capabilities of Groq to provide real-time, context-sensitive form automation directly within the browser.

\section{System Architecture}
The system comprises three core components: a Chrome Extension (Manifest V3), a Python-based backend API, and a CI/CD infrastructure for deployment.

\subsection{Chrome Extension Frontend}
The frontend is built using standard web technologies (HTML, CSS, JavaScript) and follows the Manifest V3 architecture.
\begin{enumerate}
    \item \textbf{Content Script}: Injects a subtle trigger button into every input field on the DOM. It utilizes a \texttt{MutationObserver} to handle dynamic content loading (SPAs).
    \item \textbf{Popup UI}: Provides a user interface for profile management and resume uploads.
    \item \textbf{Service Worker}: Handles background tasks and communication between the content script and the backend API.
\end{enumerate}

\subsection{FastAPI Backend}
The backend is a high-performance API built with FastAPI (Python). It handles:
\begin{itemize}
    \item \textbf{PDF Parsing}: Extraction of text from uploaded resumes using \texttt{PyPDF2}.
    \item \textbf{Prompt Engineering}: Constructing context-rich prompts for the LLM, combining user profile data with form field metadata (label, placeholder, surrounding text).
    \item \textbf{LLM Integration}: Communicating with the Groq API to generate responses.
\end{itemize}

\section{Implementation Details}
\subsection{DevOps Pipeline}
To ensure reliability and scalability, the project implements a robust DevOps pipeline:
\begin{itemize}
    \item \textbf{Containerization}: A multi-stage Dockerfile optimizes the backend image size.
    \item \textbf{Infrastructure as Code}: Terraform prompts AWS EC2 instances and security groups.
    \item \textbf{Configuration Management}: Ansible playbooks automate the server setup and application deployment.
    \item \textbf{Orchestration}: Helm charts are provided for Kubernetes deployment.
\end{itemize}

\subsection{Context-Aware Autofill Algorithm}
When a user triggers the autofill, the content script captures the target element's HTML attributes. The backend constructs a prompt:
\begin{quote}
"Given the user profile [Profile Data] and the form field [Field Label], generate a suitable response."
\end{quote}
This prompt is sent to the Groq API, and the response is inserted into the field.

\section{Conclusion}
The \textit{AI Form Filler} demonstrates the practical application of LLMs in automating everyday web tasks. By combining a lightweight browser extension with a powerful backend and modern DevOps practices, we provide a scalable solution to the problem of manual data entry. Future work will focus on handling multi-step forms and improving privacy through local LLM inference.

\bibliographystyle{IEEEtran}
\bibliography{references}

\end{document}
